\documentclass[letterpaper]{article}
\usepackage{natbib,alifeconf}
\usepackage{hyperref}

\title{Several small reserves promote the evolutionary generation and maintenance of diversity }
\author{Emily Dolson$^{1,2,3}$, Michael Wiser$^{1,3,4}$ \and Charles Ofria$^{1,2,3}$ \\
\mbox{}\\
$^1$BEACON Center for the Study of Evolution in Action  \\
$^2$Computer Science and Engineering, Michigan State University, East Lansing, MI, USA \\
$^3$Ecology, Evolutionary Biology, and Behavior, Michigan State University, East Lansing, MI, USA\\
$^4$Integrative Biology, Michigan State University, East Lansing, MI, USA\\
dolsonem@msu.edu}


\begin{document}
\maketitle

\begin{abstract}
This paper describes the formatting requirements for papers submitted to the 15th International Conference on the Synthesis and Simulation of Living Systems (ALIFEXV). MIT Press will publish the proceedings in a single online open-access volume.  The proceedings will be assembled directly from Portable Document Format (PDF) files furnished by the authors. To ensure that all articles in the on-line publication have a uniform appearance, authors must adhere to the following instructions. The publication style used here is identical to the style of the ALIFE 14 conference to ensure uniformity over both conferences.
\end{abstract}

\section{Introduction}

Protecting biodiversity is generally acknowledged to be an important conservation goal for a number of reasons, including its role in maintaining various ecosystem services (e.g. carbon sequestration), and its potential as a reservoir of useful and undiscovered genetic innovations (Gaston and Spicer, 2004; Hassan et al., 2005; Loreau et al., 2001; Montoya et al., 2012). However, there is another reason that biodiversity is critically important, which is often overlooked - in order for evolution to occur, there must be a standing pool of diversity. Since adaptation to new environments will be a critical component of the long-term survival of many lineages in the face of climate change, it is important to consider conservation of biodiversity in the context of evolution (Cowling and Pressey, 2001; Mace and Purvis, 2008; Smith et al., 2014). 

	Most conservation biology research requires a broad spatial scale. Incorporating the long temporal scale required to study evolution makes this already challenging problem intractable in most cases. As a result, very little research on conservation schemes to date has taken evolution into account (but see(Evolutionary Conservation Biology, 200)). Artificial Life techniques such as digital evolution have a lot of potential as an approach to overcoming these obstacles; they allow for the formation of interesting ecologies in a system with a fast enough generation time to do large scale evolution experiments.
    
    Here, we will use digital evolution to revisit the single-large vs. several small (SLOSS) debate from a perspective that incorporates evolutionary theory. The SLOSS debates emerged from the theory of island biogeography (MacArthur and Wilson, 1967). The original argument was that, since large islands have more species, larger reserves should be better for conserving biodiversity (Diamond, 1975). However, this effect is counterbalanced by the fact that placing more reserves might result in sampling from multiple different species pools (Simberloff and Abele, 1976). More recent refinements have considered the placement of the reserves relative to each other and interconnectivity between them (Saunders et al., 1991; Tjørve, 2010). 
    
	Evolutionary dynamics likely add additional weight to the argument for several small reserves for a number of reasons. First, transient fitness gains can result in a single lineage sweeping a reserve relatively quickly and wiping out standing diversity. Second, separating reserves decreases the colonization rate, giving other lineages time to gain beneficial mutations of their own (“Island Models,” 2005; Whitley et al., 1998). Third, in sexually reproducing populations, spatial isolation can increase the likelihood of speciation.
    
	Many factors interact to bring about the complex spatial eco-evolutionary dynamics that we observe in biological ecosystems. Indeed, the interactions between various factors are a large part of the reason that the relative benefits of different reserve placement strategies have been so hard to untangle. Here, we seek only to lay the groundwork for addressing the impact of evolution on these questions. In order to facilitate this, we will deal with the simplest possible case: a population of sessile, asexual organisms at the same trophic level. Movement, sexual recombination, and predation likely have dramatic impacts on the resulting dynamics, However, in order to understand these effects, we must first understand the behavior of a system without them. Additionally, for the purposes of this paper, we assume an entirely homogeneous environment, eliminating the possibility for complex interactions among habitat heterogeneity, species diversity, and reserve area (Kadmon and Allouche, 2007).

\section{Methods}

We conducted our experiments in silico, using the Avida Digital Evolution Platform (Ofria and Wilke, 2004). The world of Avida is a two-dimensional grid of cells occupied by “digital organisms.” These organisms are actually computer programs; their genomes are sequences of simple computer instructions. At the beginning of the experiment, we seed the world with a single ancestor that contains the instructions necessary to copy itself. As organisms copy themselves, they periodically make mistakes, introducing mutations. Some of these mutations will improve the efficiency of self-replication, so the organisms that have them will copy themselves faster than the others. If there is no space available for an organism’s offspring, the offspring will replace an existing organism. As a result, there is selection for organisms that can replicate themselves faster. Because there are mutation, inheritance, and selection, evolution by natural selection occurs. 

	To allow for the formation of more complex ecologies, we can also choose to reward organisms for performing various computational tasks by allowing them to execute their genomes faster. These tasks can be thought of as pathways for metabolizing various resources, and allow for different organisms to specialize on different survival strategies. To allow for the formation of a stable ecosystem, we can establish negative frequency dependence by linking each task to a limited resource, such that organisms are rewarded for a task in proportion to the amount of the relevant resource that they have access to, (Chow et al., 2004). 
    
	For this study, we started by evolving ten populations in the limited resource environment described above. Each population was started from the same hand-coded self-replicator, but was then allowed to diverge for 100,000 updates, a length of time roughly equivalent to 2000 generations. We then placed these populations in one of nine environments for 100,000 more updates. Each environment had reserves placed across it in a different configuration. All environments had a total of 900 of the 3600 grid cells placed in square reserves that tiled evenly across the environment. Reserve configurations varied from having many very small reserves (900 1x1 reserves at the extreme) to having a single very large reserve (one 30x30 reserve at the extreme). Because world size was held constant, configurations with more reserves necessarily involved them being placed closer together (there was always one reserve's worth of space between reserves on all sides). Organisms living in areas outside of the reserves were at risk of being randomly killed each update, according to a specified kill rate (0, 100, 200, 300, 400, or 500 cells selected per update). In order to ascertain what effect allowing populations to evolve was having on our results, we also ran a series of controls in which mutations were turned off for the second 100,000 updates. We ran 10 replicates per treatment in a fully factorial design across initial population, environment, five kill rates, and mutations being turned on vs. off.
    
	There are three possible mechanisms by which our reserve schemes could drive changes in diversity over evolutionary time - a reserve configuration might: 1) sample from a different range of locations across the environment, 2) promote improved maintenance of existing diversity, and/or 3)  promote improved generation of new diversity. To address possible mechanism (1), we measured the number of phenotypes in any reserve at the beginning of the experiment for each condition. A related effect  that may interact with this mechanism is that different reserve configurations might change the way that phenotypes are distributed across the world, making them more or less clumped, and so change the relative importance of sampling across the environment. We accounted for this by measuring spatial autocorrelation in phylogenetic depth with Moran's I as implemented in the APE R package (Paradis et al., 2004) for each condition. To address possible mechanisms (2) and (3), we collected a variety of data on the phenotype-area relationships in our data: phenotype richness within each reserve, total phenotype richness captured across all reserves in a replicate, count of phenotypes lost over the experimental treatment, and count of new phenotypes that evolved over the course of the experimental treatment.
    
	All analyses were conducted using the R Statistical Computing Language, version 3.2.3 (Team, 2013). To account for non-independence introduced by our nested design, significance of relationships between variables was determined using linear mixed models with random effects for initial population and replicate where appropriate, as implemented in the lme4 R package (Bates et al., 2015).

\section{Results}

Overall, the replicates in which mutations were allowed during the second 100,000 updates had substantially higher diversity (both richness and Shannon entropy) at the end of the experiment than replicates for which mutations were turned off (see Fig. 1) (Chi-squared = 10123, p <.0001). Ecology mode runs had an average of 192.009 +- 1.345 fewer phenotypes remaining at the end compared with …. For the evolutionary runs, having a greater number of smaller reserves resulted in higher diversity at the end of the experiment (Chi-squared = 1172.6, p <.0001, effect=0.0956+- 0.0026 log(phenotype count) per additional log(patch count) units). This effect was present but substantially weaker among ecology mode runs (Chi-squared = 25.298, p <.0001, effect=0.0084 +- 0.0017 log(phenotype count) per additional log(patch count) units). 

	Moran's I of the final phylogenetic depths in the population was highly variable among ecology mode runs, likely because  they had so few phenotypes remaining at the end. Among runs in which mutations were allowed, variance in Moran's I was substantially lower within each condition. The average value for Moran's I increased from the average value for initial populations for configurations with large reserves and decreased for configurations with small reserves (see Fig. 2).
    
	In the 10 initial populations, the relationship between phenotype richness within a patch and the size of that patch followed the pattern of a standard species-area relationship (see Fig. 3), with a linear relationship between the logarithms  of patch size and patch richness. Despite this positive relationship, the total phenotypic richness summed across all patches within an environment was negatively correlated with the area of those patches. This negative relationship strengthened when populations were allowed to evolve for 100,000 updates, while the slope of the phenotype-area relationship decreased slightly (see Fig. 4). These effects both weakened dramatically in the ecology mode control, but remained present.
    
	Digging into the drivers behind the negative relationship between patch area and phenotypic richness among the evolutionary condition, we now examine gain and loss of phenotypes during the second 100,000 updates of the experiment. First, we measured the number of phenotypes that were present in the initial population but not in the final population, i.e. “lost phenotypes.” Among the ecology mode runs, phenotype loss was relatively constant across different reserve placements. Among the evolutionary runs, however, there was a positive relationship between phenotype loss and patch sizes – environments with larger patches resulted in greater phenotype extinction by the end of the experiment (see Fig. 5). We also measured the number of phenotypes that were not present in the initial population but were present in the final population, i.e. newly evolved phenotypes. Among the evolutionary runs, the count of newly evolved phenotypes had a negative relationship with patch size – environments with larger patches resulted in fewer newly evolved phenotypes.

\section{Discussion}

Allowing our populations to evolve dramatically increased the extent to which many smaller patches are favored over fewer larger patches if the goal is maximizing phenotypic richness at the end. This result stems from three primary factors. First, having more small reserves incorporates a greater number of phenotypes, likely due to to the substantial spatial autocorrelation. However, the impact of this factor might be diminished by the fact that spatial autocorrelation decreased over evolutionary time in environments with many small reserves. 

	Second, smaller patches do a better job of maintaining existing diversity – fewer phenotypes from the initial population go extinct in these conditions. There are a number of potential evolutionary drivers behind this dynamic. The most likely is the fact that having smaller patches decreases the ease with which slightly better-adapted phenotypes can sweep through the population, out-competeing everything else.  Instead, lineages that may have been slightly less fit after the first 100,000 updates have time to get beneficial mutations before they come up against the current best phenotypes.
    
	Third, smaller patches do a better job of generating new diversity – more new phenotypes appear in these conditions. This dynamic is likely driven by factors to those that drive the previous dynamic. When organisms are competing against fewer other organisms for space, the selection pressure is weakened. This is generally believed to allow time for evolutionary innovation and increased diversification into new niches. On the flip side, it also means that the population can support more organisms that are not particularly well adapted. It would be interesting to explore the effect of these conditions on overall population fitness.
    
	Whereas these results have potentially important implications for management practice, it's important to recognize that a number of our simplifying assumptions will bias our results against fewer larger patches. None of the organisms in our experiment are, for instance, larger or at a higher trophic level than any other organisms. In nature, some organisms require vastly more space than others in order to have access to sufficient energy sources. These results represent the first layer in an understanding of how evolution interacts with reserve placement. Other important topics for future research include the effect of corridors, distance between reserves (as a separate variable from size and count), placement of reserves in relation to spatial resources, interactions with motile organisms, and the impact of sexual recombination and gene flow on diversity in these systems. 

\section{Acknowledgements}

We extend our thanks to Phoebe Zarnetske for her guidance on the ecological theory and spatial analysis in this paper. This research has been supported in part by the National Science Foundation (NSF) BEACON Center under Cooperative Agreement DBI-0939454, by the National Science Foundation Graduate Research Fellowship under Grant No. DGE-1424871, and by Michigan State University through computational resources provided by the Institute for Cyber-Enabled Research. Any opinions, findings, and conclusions or recommendations expressed in this material are those of the author(s) and do not necessarily reflect the views of the NSF.


\bibliographystyle{apalike}
\bibliography{example}

\end{document}
